% Options for packages loaded elsewhere
\PassOptionsToPackage{unicode}{hyperref}
\PassOptionsToPackage{hyphens}{url}
\PassOptionsToPackage{dvipsnames,svgnames,x11names}{xcolor}
%
\documentclass[
  letterpaper,
  DIV=11,
  numbers=noendperiod]{scrartcl}

\usepackage{amsmath,amssymb}
\usepackage{iftex}
\ifPDFTeX
  \usepackage[T1]{fontenc}
  \usepackage[utf8]{inputenc}
  \usepackage{textcomp} % provide euro and other symbols
\else % if luatex or xetex
  \usepackage{unicode-math}
  \defaultfontfeatures{Scale=MatchLowercase}
  \defaultfontfeatures[\rmfamily]{Ligatures=TeX,Scale=1}
\fi
\usepackage{lmodern}
\ifPDFTeX\else  
    % xetex/luatex font selection
\fi
% Use upquote if available, for straight quotes in verbatim environments
\IfFileExists{upquote.sty}{\usepackage{upquote}}{}
\IfFileExists{microtype.sty}{% use microtype if available
  \usepackage[]{microtype}
  \UseMicrotypeSet[protrusion]{basicmath} % disable protrusion for tt fonts
}{}
\makeatletter
\@ifundefined{KOMAClassName}{% if non-KOMA class
  \IfFileExists{parskip.sty}{%
    \usepackage{parskip}
  }{% else
    \setlength{\parindent}{0pt}
    \setlength{\parskip}{6pt plus 2pt minus 1pt}}
}{% if KOMA class
  \KOMAoptions{parskip=half}}
\makeatother
\usepackage{xcolor}
\setlength{\emergencystretch}{3em} % prevent overfull lines
\setcounter{secnumdepth}{-\maxdimen} % remove section numbering
% Make \paragraph and \subparagraph free-standing
\makeatletter
\ifx\paragraph\undefined\else
  \let\oldparagraph\paragraph
  \renewcommand{\paragraph}{
    \@ifstar
      \xxxParagraphStar
      \xxxParagraphNoStar
  }
  \newcommand{\xxxParagraphStar}[1]{\oldparagraph*{#1}\mbox{}}
  \newcommand{\xxxParagraphNoStar}[1]{\oldparagraph{#1}\mbox{}}
\fi
\ifx\subparagraph\undefined\else
  \let\oldsubparagraph\subparagraph
  \renewcommand{\subparagraph}{
    \@ifstar
      \xxxSubParagraphStar
      \xxxSubParagraphNoStar
  }
  \newcommand{\xxxSubParagraphStar}[1]{\oldsubparagraph*{#1}\mbox{}}
  \newcommand{\xxxSubParagraphNoStar}[1]{\oldsubparagraph{#1}\mbox{}}
\fi
\makeatother


\providecommand{\tightlist}{%
  \setlength{\itemsep}{0pt}\setlength{\parskip}{0pt}}\usepackage{longtable,booktabs,array}
\usepackage{calc} % for calculating minipage widths
% Correct order of tables after \paragraph or \subparagraph
\usepackage{etoolbox}
\makeatletter
\patchcmd\longtable{\par}{\if@noskipsec\mbox{}\fi\par}{}{}
\makeatother
% Allow footnotes in longtable head/foot
\IfFileExists{footnotehyper.sty}{\usepackage{footnotehyper}}{\usepackage{footnote}}
\makesavenoteenv{longtable}
\usepackage{graphicx}
\makeatletter
\def\maxwidth{\ifdim\Gin@nat@width>\linewidth\linewidth\else\Gin@nat@width\fi}
\def\maxheight{\ifdim\Gin@nat@height>\textheight\textheight\else\Gin@nat@height\fi}
\makeatother
% Scale images if necessary, so that they will not overflow the page
% margins by default, and it is still possible to overwrite the defaults
% using explicit options in \includegraphics[width, height, ...]{}
\setkeys{Gin}{width=\maxwidth,height=\maxheight,keepaspectratio}
% Set default figure placement to htbp
\makeatletter
\def\fps@figure{htbp}
\makeatother

\KOMAoption{captions}{tableheading}
\makeatletter
\@ifpackageloaded{caption}{}{\usepackage{caption}}
\AtBeginDocument{%
\ifdefined\contentsname
  \renewcommand*\contentsname{Table of contents}
\else
  \newcommand\contentsname{Table of contents}
\fi
\ifdefined\listfigurename
  \renewcommand*\listfigurename{List of Figures}
\else
  \newcommand\listfigurename{List of Figures}
\fi
\ifdefined\listtablename
  \renewcommand*\listtablename{List of Tables}
\else
  \newcommand\listtablename{List of Tables}
\fi
\ifdefined\figurename
  \renewcommand*\figurename{Figure}
\else
  \newcommand\figurename{Figure}
\fi
\ifdefined\tablename
  \renewcommand*\tablename{Table}
\else
  \newcommand\tablename{Table}
\fi
}
\@ifpackageloaded{float}{}{\usepackage{float}}
\floatstyle{ruled}
\@ifundefined{c@chapter}{\newfloat{codelisting}{h}{lop}}{\newfloat{codelisting}{h}{lop}[chapter]}
\floatname{codelisting}{Listing}
\newcommand*\listoflistings{\listof{codelisting}{List of Listings}}
\makeatother
\makeatletter
\makeatother
\makeatletter
\@ifpackageloaded{caption}{}{\usepackage{caption}}
\@ifpackageloaded{subcaption}{}{\usepackage{subcaption}}
\makeatother

\ifLuaTeX
  \usepackage{selnolig}  % disable illegal ligatures
\fi
\usepackage{bookmark}

\IfFileExists{xurl.sty}{\usepackage{xurl}}{} % add URL line breaks if available
\urlstyle{same} % disable monospaced font for URLs
\hypersetup{
  pdftitle={final\_project},
  pdfauthor={Leslie Cervantes Rivera \& Kevin Le},
  colorlinks=true,
  linkcolor={blue},
  filecolor={Maroon},
  citecolor={Blue},
  urlcolor={Blue},
  pdfcreator={LaTeX via pandoc}}


\title{final\_project}
\author{Leslie Cervantes Rivera \& Kevin Le}
\date{}

\begin{document}
\maketitle


\section{Background}\label{background}

The Infant Mortality and Fertility Rates Dataset provides data on infant
mortality rates (measured as the number of infant deaths per 1,000 live
births) and fertility rates (measured as the number of births per 1,000
women per year) across U.S. states. The Infant Mortality Dataset covers
the years 2003 to 2023, while the Fertility Rates Dataset spans 2016 to
2023.

Infant mortality refers to the death of baby that occurs before their
first birthday. It is a key indicator of a population's overall health,
reflecting the social, economic, and healthcare conditions within a
state. High infant mortality rates could indicate insufficient
healthcare access or inequalities in medical services, particularly in
low-income and rural areas. Over the years, improvements in screening
and treatment for illnesses, better obstetric management, and neonatal
care have contributed in the declining of infant mortality rates, but
disparities still exist within state and demographic groups.
(\url{https://www.vcuhealth.org/news/infant-mortality-rates-declining-but-sudden-unexpected-infant-death-is-on-the-rise/\#:~:text=Wolf\%20attributes\%20declining\%20overall\%20infant,media\%20on\%20infant\%20sleep\%20practices.})
(\url{https://www.nichd.nih.gov/health/topics/infant-mortality/topicinfo})

Fertility rate represents the total number of children a woman has
during her reproductive years. It plays a crucial role in population
growth and demographical planning, influencing services such as
education, healthcare, and workforce development. From 2007 to 2022, the
fertility rate has dropped by about 19\%, influenced by the health of
the economy, social, health, and historical events. These changes have
long-term implications, as it leads to an aging population, smaller
workforce, and economic strain on government budgets.
(\url{https://www.britannica.com/topic/fertility-rate})
(\url{https://www.northwell.edu/news/the-latest/us-fertility-rate-decline-impact\#:~:text=After\%20a\%20few\%20decades\%20of,historical\%20events\%E2\%80\%94affect\%20family\%20sizes.})

The Population Demographics dataset provides data on the years 2016 to
2023, including total population of sex and age, marital status,
educational attainment, SNAP, and health insurance coverage based on
ethnicity.

\subsection{Data Sources}\label{data-sources}

\subsubsection{Birth Rates and Infant Mortality
Datasets}\label{birth-rates-and-infant-mortality-datasets}

The datasets come from the Centers for Disease Control and Prevention
(CDC), collected from anywhere a person receives healthcare. This may
lead to limitations as it is up to the city, county, and state to decide
what information is collected, and how and when it can be shared by the
CDC.
(\url{https://archive.cdc.gov/www_cdc_gov/surveillance/data-modernization/basics/where_does_our_data_come_from.html})

median income -\textgreater{} 2020 reasoning
\url{https://www.census.gov/data/developers/data-sets/acs-1year.2020.html\#list-tab-843855098}

\subsubsection{Population Demographics}\label{population-demographics}

The dataset comes from the United States Census Bureau under the
category American Community Survey named Selected Population Profile in
the United States. The Census Bureau collects their data from
respondents directly through censuses and surveys. Primary sources come
from administrative data.

\section{Question of Interest}\label{question-of-interest}

\section{Exploratory Data Analysis
(EDA)}\label{exploratory-data-analysis-eda}

\subsection{}\label{section}

\begin{verbatim}
[1] "Hello"
\end{verbatim}

\section{Linear Regression}\label{linear-regression}

\[
\text{total_fer}
    = \beta_0 + \underbrace{\beta_1\left(\text{less hs}\right)_i + \cdots + \beta_5\left(\text{grad}\right)_i}_{\text{education attainment}} 
    + \underbrace{\beta_6\text{income_median}_i + \beta_7\text{SNAP}_i}_{\text{economic indicators}}
    + \underbrace{\beta_8\text{private}_i + \cdots + \beta_{10}\text{no coverage}_i}_{\text{health coverage}}
    + \underbrace{\beta_{11}\text{total_mort}_i}_{\text{infant mortality}} + \epsilon_i
\]




\end{document}
